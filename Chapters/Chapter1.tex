% Chapter 1
%\title  {Weather Classification}
\chapter{Introduction} % Write in your own chapter title
\label{Chapter1}
\lhead{Chapter 1. \emph{Introduction}} % Write in your own chapter title to set the page header


\section{Overview}

The computer is one of the most significant inventions in history. It provides huge power in the data processing field, like computer vision. Also, computer systems can aid people in daily life, for example driverless vehicles. However, the human brain still has compelling advantage in some fields, like identifying our keys in our pocket by feel. The complex processes of taking in raw data and taking action based on the pattern are regarded as pattern recognition. Pattern recognition has been important for people in daily life for a long period and the human brain has developed an advanced neural and cognitive system for such tasks.

Weather classification is one of the most important pattern recognition tasks which relates to our work and lives. There are several major kinds of weather conditions, like sunny, cloudy, rainy. And people can classify them easily through their eyes. However, it is not an easy job for machines, especially in literature on computer vision. 

In this thesis, we describe an approach to this problem of weather classification based upon the major big trend in machine learning, and more precisely, deep learning. It differs from the feature detectors method, that extracts features manually, and then trains a model to do classification.  

Compared to shadow learning which includes decision trees, the SVM and naive bayes, deep learning passes input data through several non-linearity functions to generate features and does classification based on those features. It generates a mapping and finds the optimal solution.

\section{Statistical Pattern Recognisation}

In the statistical approach, the pattern is represented in terms of d dimensional feature vectors and each element of the vector describes some subjects of the example. In brief, three components are essential to do statistical pattern recognization. First is a representation of the model. Second is an objective function used to evaluate the accuracy of the model. Third is an optimisation method for learning a model with minimum errors.

\section{Artificial Neural Networks(ANNs)}

Artificial neural networks were proposed in the mid-20th century. The term is inspired by the structure of neural networks in the brain. It is one of the most successful statistical learning models used to approximate functions. Learning with ANNs yields an effective learning paradigm and has achieved cutting-edge performance in object classification. 

The single-layer ANNs  has shown great success for a simple model. For increasing complex models and applications, multi-layer ANNs has exhibited the power of features learning. With hardware developing at the same time, the demand for more efficient optimising and model evaluation methods has increased promptly. 

Recent development in ANNs has greatly advanced the performance of state-of-the-art visual recognition systems. With implementing deep convolutional neural networks, ANNS achieves top accuracy in the ImageNet Challenge. The model has been used in related fields and performs well in pattern recognisation.

\section{Weather Classification}

Weather classification is an important job in daily life. In this thesis, we are focused on two classes of weather conditions, sunny and cloudy. 

There are some obstacles in front of weather classification. First, because the number of inputting pixels is high, say a 5$00 \times 500$ RPG image means $750,000$ pixels, computation is expensive. Second, some simple middle level image characters are difficult to be recognised by machines, like light, shading, sun shine. It is still not easy to detect these characters with high accuracy.  Third, there are no decisive features, for example brightness and lightness, to classify scenes. For example, sun shine can be both found in sunny and cloudy weather. Last but not least, outdoor images have various background.


