% Chapter 1
%\title  {Weather Classification}
\chapter{Introduction} % Write in your own chapter title
\label{Chapter1}
\lhead{Chapter 1. \emph{Introduction}} % Write in your own chapter title to set the page header


\section{Overview}

The computer is one of the most significant inventions in history. It provides huge power in the data processing field, such as computer vision. Also, computer systems can aid people in daily life, for example driverless vehicles. However, the human brain still has compelling advantage in some fields, such as identifying our keys in our pocket. The complex processes of taking in raw data and taking action based on the pattern are regarded as pattern recognition. Pattern recognition has been important for people in daily life for a long period and the human brain has developed an advanced neural and cognitive capacity for such tasks.

Weather classification is one of the most important pattern recognition tasks which relates to our work and lives. There are several major kinds of weather conditions, such as sunny, cloudy, rainy. Human can classify them easily through their eyes. However, it is not an easy task for machines, especially in computer vision. 

In this thesis, we describe an approach to this problem of weather classification based upon the major trend in machine learning, and more precisely, deep learning. It differs from those conventional methods, which extract features manually, and then train a model to perform classification.

Compared to shallow learning which includes decision trees, SVM and naive bayes, deep learning passes input data through several non-linearity functions to generate features and performs classification based on those features. It generates a mapping and finds the optimal solution.

\section{Statistical Pattern Recognisation}

In the statistical approach, a pattern is represented in terms of $d$-dimensional feature vectors and each element of the vector describes some subjects of the example. In summary, three components are essential for statistical pattern recognisation. The first one is a representation of the model. The second one is an error function used to evaluate the performance of the model. The third one is an optimisation method for learning a model.

\section{Artificial Neural Networks (ANNs)}

Artificial neural networks were proposed in the mid-20th century. The term is inspired by the structure of neural networks in the human brain. It is one of the most successful statistical learning models used to approximate functions. Learning with ANNs yields an effective learning paradigm and has achieved cutting-edge performance in object classification. 

Single-layer ANNs  have shown great success for a simple model. For increasingly complex models and applications, multi-layer ANNs have exhibited the power of features learning. With hardware being developing at the same time, the demand for more efficient optimising and model evaluation methods has increased promptly. 

Recent development in ANNs has greatly advanced the performance of state-of-the-art visual recognition systems. With implementing deep convolutional neural networks, ANNs achieved top accuracy in the ImageNet Challenge. The model has been used in related fields and performs well in pattern recognition.

\section{Weather Classification}

Weather classification is an important task in daily life. In this thesis, we focus on two classes of weather conditions, sunny and cloudy. 

There are some obstacles in front of weather classification. Firstly, because the number of inputting pixels is high, for example a $500 \times 500$ RPG image containing $750,000$ pixels, computation is expensive. Secondly, some simple middle level image characters are difficult to be recognised by machines, such as light, shading, and sun shine. It is still not easy to detect these characters with high accuracy. Thirdly, there are no decisive features, such as brightness and lightness, to classify scenes. Sun shine can be both found in sunny and cloudy weather. Last but not least, outdoor images have various backgrounds.


