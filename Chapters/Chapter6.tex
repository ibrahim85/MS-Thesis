% Chapter 6

\chapter{Background} % Write in your own chapter title
\label{Chapter6}
\lhead{Chapter 6. \emph{Background}}

Compared to general classification, multi-label classification has different evaluation metrics and learning algorithms.

\section{Evaluation Metrics}

In supervised learning, different metrics, like accuracy and area under the ROC curve, are used to evaluate the generalisation performance of a model. In the multi-label learning, evaluating performance is more complicated  than single-label classification problems because of increasing number of labels simultaneously. Therefore, two main types of evaluation methods are implemented in multi-label learning, say example-based metrics\citep{ghamrawi2005collective} and label-metrics\citep{tsoumakas2007random}.

The two types of metrics evaluate outputs of classifiers from different perspectives. Given that $S = {(x_{i}, Y_{i})}$ is a test sample and $h(\cdot)$ is the learned multi-label classifier. The concept of example-based metrics is achieve all class labels of each test sample, and then compute the mean value of test sets to evaluate generalisation performance. Compared with considering all class labels simultaneously, label-based metrics evaluate performance by treating each class label separately and computing macro/micro-averaged value of all class labels.

In a supervised classification problem, there are a ground truth output and a predicted output for a test sample. So the results of each test sample can be assigned to one of the four categories:
\begin{itemize}
\item True Positive (TP) - label is positive and prediction is also positive
\item True Negative (TN) - label is negative and prediction is also negative
\item False Positive (FP) - label is negative but prediction is positive
\item False Negative (FN) - label is positive but prediction is negative
\end{itemize}

Here we define a set $D$ of $N$ examples and $Y_{i}$ to be a family of ground truth label sets and $P_{i} = h(x_{i})$ to be a family of predicted label set. The set of all unique labels is
\begin{equation}\label{eq:UniLabel}
L = \bigcup_{i=0}^{N-1} L_{i}
\end{equation}
While the definition of indicator function $I_{A}$ on a set $A$ is presented:
\begin{equation}\label{eq:IndicatorFunc}
I_{A}(x) =
  \begin{cases}
    1       & \quad \text{if } x \in A\\
    0  & \quad \text{otherwise}\\
  \end{cases}
\end{equation}

\subsection{Example-based Metrics}

\textbf{Hamming Loss} evaluates performance via counting the number of misclassification labels. The smaller value of hamming loss is, the better performance the model has.
\begin{equation}\label{eq:HammingLoss}
\frac{1}{N \cdot \left|L\right|} \sum_{i=0}^{N - 1} \left|L_i\right| + \left|P_i\right| - 2\left|L_i
          \cap P_i\right|
\end{equation}

\textbf{Subset Accuracy} evaluates the fraction of correctly predicted example while the predicted label set is identical to the ground truth label set. It is equivalent to traditional accuracy metric.
\begin{equation}\label{eq:SubsetAcu}
\frac{1}{N} \sum_{i=0}^{N-1} I_{\{L_i\}}(P_i)
\end{equation}

\textbf{Precision} is defined:
\begin{equation}\label{eq:Precision}
\frac{1}{N} \sum_{i=0}^{N-1} \frac{\left|L_i \cap P_i\right|}{\left|P_i\right|}
\end{equation}

\textbf{Recall} is defined:
\begin{equation}\label{eq:Recall}
\frac{1}{N} \sum_{i=0}^{N-1} \frac{\left|L_i \cap P_i\right|}{\left|L_i\right|}
\end{equation}

\textbf{Accuracy} is defined as
\begin{equation}\label{eq:Accuracy}
\frac{1}{N} \sum_{i=0}^{N - 1} \frac{\left|L_i \cap P_i \right|}
        {\left|L_i\right| + \left|P_i\right| - \left|L_i \cap P_i \right|}
\end{equation}

\textbf{F1 Measure} is an integrated version combined by harmonic mean of \textbf{Precision} and \textbf{Recall}. 
\begin{equation}\label{eq:F1Measure}
\frac{1}{N} \sum_{i=0}^{N-1} 2 \frac{\left|P_i \cap L_i\right|}{\left|P_i\right| \cdot \left|L_i\right|}
\end{equation}

\subsection{Label-based Metrics}

\textbf{Macro Precision} (precision averaged across all labels) is defined as
\begin{equation}\label{eq:MacroPrecision}
PPV(\ell)=\frac{TP}{TP + FP}=
          \frac{\sum_{i=0}^{N-1} I_{P_i}(\ell) \cdot I_{L_i}(\ell)}
          {\sum_{i=0}^{N-1} I_{P_i}(\ell)}
\end{equation}

\textbf{Macro Recall} (recall averaged across all labels) is defined as
\begin{equation}\label{eq:MacroRecall}
TPR(\ell)=\frac{TP}{P}=
          \frac{\sum_{i=0}^{N-1} I_{P_i}(\ell) \cdot I_{L_i}(\ell)}
          {\sum_{i=0}^{N-1} I_{L_i}(\ell)}
\end{equation}

\textbf{F1 Measure by label} is the harmonic mean between \textbf{Precision} and \textbf{Recall}. 
\begin{equation}\label{eq:LabelAccuracy}
F1(\ell) = 2
                            \cdot \left(\frac{PPV(\ell) \cdot TPR(\ell)}
                            {PPV(\ell) + TPR(\ell)}\right)
\end{equation}

\textbf{Micro Precision} (precision averaged over all example/label pairs) is defined as
\begin{equation}\label{eq:MicroPrecision}
\frac{TP}{TP + FP}=\frac{\sum_{i=0}^{N-1} \left|P_i \cap L_i\right|}
          {\sum_{i=0}^{N-1} \left|P_i \cap L_i\right| + \sum_{i=0}^{N-1} \left|P_i - L_i\right|}
\end{equation}

\textbf{Micro Recall} (recall averaged over all the example/label pairs) is defined as
\begin{equation}\label{eq:MicroRecall}
\frac{TP}{TP + FN}=\frac{\sum_{i=0}^{N-1} \left|P_i \cap L_i\right|}
        {\sum_{i=0}^{N-1} \left|P_i \cap L_i\right| + \sum_{i=0}^{N-1} \left|L_i - P_i\right|}
\end{equation}

\textbf{Micro F1 Measure by label} is the harmonic mean between \textbf{Micro Precision} and \textbf{Micro Recall}. 
\begin{equation}\label{eq:LabelMicroAccuracy}
2 \cdot \frac{TP}{2 \cdot TP + FP + FN}=2 \cdot \frac{\sum_{i=0}^{N-1} \left|P_i \cap L_i\right|}{2 \cdot
        \sum_{i=0}^{N-1} \left|P_i \cap L_i\right| + \sum_{i=0}^{N-1} \left|L_i - P_i\right| + \sum_{i=0}^{N-1}
        \left|P_i - L_i\right|}
\end{equation}

For the label-based metrics, the larger the metrics value means the higher generalisation performance.

With the previous metrics, there are diverse methods to evaluate model generalization performance. In most multi-label classification, learning algorithms optimise one of the metrics. To make evaluation fair and precise, learning algorithms should be tested on different metrics to evaluate performance.

Most metrics are non-convex and discrete, and most algorithms turn to optimise alternative multi-label metrics. Recently, some researcher study the consistency of multi-label learning\citep{gao2013consistency}. 

\section{Learning Algorithms}

Algorithms play a key role in the literature on machine learning, and there is no exception in multi-label learning. The capability of representation is important to evaluate the performance of an algorithm. Moreover, some relating criterions can be used to measure performance. First, the broad spectrum should be considered. An algorithm should cover a range of algorithmic design strategies with unique characteristics. Second, it is reasonable to evaluate the impact which the algorithm poses on the multi-label learning settings. Last, computational complexity is a critical factor to evaluate if an algorithm is practical.

\subsection{Problem Transformation Methods}

\subsubsection{Binary Relevance (BR)}

The Binary Relevance method is an elementary algorithm which decomposes a multi-label learning problem into a set of independent binary classification problems, and each problem harmonises one label in the set $L = {y_{1}, y_{2},...,y_{q}}$. The approach initially transforms the multi-label dataset into $q$ binary datasets $D_{y_{i}} (i = 1,2,...,q)$, where each $D_{Y_{I}}$ includes all samples of the multi-label dataset and has a single binary label to instruct if the dataset has or has no attribute to the relevant label. For example, if a sample has a positive value means that the sample owns the correlation label, otherwise, it does not own it. After transforming the dataset, a set of $q$ binary classifiers, say $H_{i}(E) (i = 1,2,...,q)$, have been built up using the respective training dataset $D_{y_{i}}$. 

\begin{equation}\label{eq:BinaryRelevance}
H = \{C_{y_{i}}(x, y_{i}) \to y'_{i} \in \{0,1\}| y \in L, i = 1,2,...,q\}
\end{equation}

To classify a test sample, the BR method outputs the collection of labels which are predicted with positive value by the independent binary classifiers.

The BR method combines computational efficiency and simple implementation. With a constant number of samples, the algorithm scales with size $q$ of label set $L$. Supposing that each classifier has complexity $f(|X|,|D|)$, The method has complexity $O(q \times f(|X|,|D|))$. With a limit number of $q$, the method learns a model quickly and has an reasonable performance.

One of the disadvantages is the limitation to label relationship information. Unless all labels are independent, the method losses the correlations among labels.

\subsubsection{Classifier Chains (CC)}

To achieve a better performance, the CC method has been introduced in multi-label classification\citep{read2011classifier}. The method transforms multi-label learning problems into a chain of binary classifiers based on label dependence.

For a set of $q$ labels $L$, the CC method learns $q$ classifiers as BR does. In addition, it links all classifiers in feature space. Given that there are a set of samples $D(x,y_{i}) (i = 1,2,...,q)$ where $y_{i}$ is a subset of labels $L$ and $x$ is the domain of dataset. The original dataset is transformed into a set of $q$ datasets in which the $j$th example is based on the previous dataset.

\begin{equation}\label{eq:ClassifierChains}
H = \{C_{y_{i}}(x, y_{1},y_{2},...,y_{i-1}) \to y'_{i} \in \{0,1\}| y \in L, i = 1,2,...,q\}
\end{equation}

Thus it forms a chain of binary classifiers $C_{1},C_{2},...,C_{q}$. Each classifier $C_{i}$ learns and predicts the binary association of label $y_{i}$ based on the dataset $x$ and all prior binary relevance predictions $y_{j}, j = 1,2,...,i-1$. The learning process starts from $C_{1}$ and follows the chain sequentially . Therefore, $C_{1}$ determines $p(y_{1}|x)$, and sequential classifiers, say $C_{2},...,C_{q}$, determine $p(y_{i}|x_{i},y_{1},...,y_{i-1})$.

The CC method propagates label information through classifiers, and latter classifiers take account of previous predictions. This method can retrieve lost information among labels partly. At the same time, the method keep advantages of BR, say computation efficiency and memory efficiency. The computational complexity of the method is close to the BR in terms of the number of labels and complexity of elemental classifiers $C_{i}$. As previous statement, the computational complexity of the BR is $O(q \times f(|X|,|D|))$ where $f(|X|,|D|))$ represents the elemental classifier. With extra computation introduced by previous labels, the computational complexity of the CC is $O(q \times f(|X|+q,|D|))$, while the computation will be unaffordable in case of $q \gg |X|$.

\subsection{Algorithm Adaptation Methods}

\subsubsection{Multi-label k-Nearest Neighbour (ML-kNN)}

The algorithm adapts k-nearest Neighbour techniques to propose multi-label data and utilise the MAP to make prediction through reasoning embodied labelling information among neighbours \citep{zhang2007ml}.

Given dataset $X$ and its label set $Y$, $y$ represents output vector for $x$ where $i$-th element $y(i)$ is positive if $i \in Y$, otherwise it is negative. Suppose that $N(x)$ is the set of neighbours of $x$ in the training dataset, we can compute a membership counting vector which represents the number of neighbours of $x$ owning the $l$-th label.

\begin{equation}\label{eq:KNNCounting}
C_{x}(i) = \sum_{a \in N(x)} y_{a}(i), i \in Y
\end{equation}

To test a new sample $t$, the algorithm sorts out its category in the training dataset by kNN $N(t)$. Let $H_{1}^l$ denotes that $t$ has the label $l$ and $H_{0}^l$ denotes that $t$ do not have the label $l$. If $j$ samples have the label $l$, say $E_{j}^l (j \in {0,1,...,k})$, the category vector $y_{t}$ is determined by the MAP principle:

\begin{equation}\label{eq:CategoryVec}
y_{t}(l) = \operatorname*{arg\,max}_{b \in \{0,1\}} P(H_{b}^l|E_{C_{t}(i) }^l) l \in Y
\end{equation}

With the Bayesian rule, \ref{eq:CategoryVec} can be represented as

\begin{equation}\label{eq:CategoryVecBay}
y_{t}(l) = \operatorname*{arg\,max}_{b \in \{0,1\}} P(H_{b}^l)P(E_{C_{t}(i) }^l|H_{b}^l) l \in Y
\end{equation}

As stated in equation \ref{eq:CategoryVecBay}, the output of a test sample $t$ is $y_{t}(l)$. It can be computed through prior probability $P(H_{b}^l) l \in Y, b \in \{0,1\}$ and the posterior probability $P(E_{j}^l|H_{b}^l) (j \in \{0,1,...,k\})$.

\subsubsection{Collective Multi-label Classifier (CML)}

Supposed that labels are highly interdependent in some scenarios, the CML explores multi-label conditional random field (CRF) classification models which learn  parameters for each pair of labels directly\citep{ghamrawi2005collective}.

In dataset $(x,Y)$, each sample has a corresponding random variables representation $(x,y)$, in which $y = (y_{1},y_{2},...,y_{q})  y_{i} \in \{-1,1\}$ is a binary label vector. If the sample contains $j$-th label, the $j$-th element of $y$ is $1$, otherwise $-1$. Then the aim is to learn the joint probability distribution $p(x,y)$.

The entropy of $(x,y)$ is represented as $H_{p}(x,y)$ and gives the distribution $p(\cdot,\cdot)$ of $(x,y)$. The principle of maximum entropy can be achieved by maximising $H_{p}(x,y)$. The fact is expressed with constrains on the expectation of function $f(x,y)$. The expected value can be estimated from the training dataset

\begin{equation}\label{eq:CMLExpect}
F_{k} = \frac{1}{m}\sum_{(x,y) \in D}f_{k}(x,y)
\end{equation}

According to the normalization constraint on $p(\cdot,\cdot)$, the optimal solution can be represented:

\begin{equation}\label{eq:CMLOptimal}
p(y|x) = \frac{1}{Z(x)}exp(\sum_{k \in K}\lambda_{k} \cdot f_{k}(x,y))
\end{equation}

Yielding to Gaussian distribution, parameters can be found in a convex log-posterior probability function. For a test sample $x$, the predicted label set follows to:

\begin{equation}\label{eq:CMLLabel}
y_{x} = \operatorname*{arg\,max}_{y} p(y|x)
\end{equation}

It is notable that the exact inference is only suitable for small label space via arg max, because it needs to reduce the search space of arg max significantly via the pruning strategy. The CML is a second-order approach.